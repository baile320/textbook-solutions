\documentclass{article}
\usepackage{amsmath}
\usepackage{tikz}
\usepackage{icomma}
\usepackage{hyperref}
\usetikzlibrary{positioning}
\newcommand\abs[1]{\left|#1\right|}
\newcommand\floor[1]{\lfloor#1\rfloor}

\begin{document}

	\title{%
  	Introduction to Graph Theory \\
  		\large by Richard Trudeau \\
   		Ch. 8 Solutions}
   		\author{Tyler Bailey}
	\maketitle

	\begin{enumerate}
		\item[1] Use Exercise 11 of Chapter 2 to prove that every graph has an even number of odd vertices.
		\item[2] Find all integers $v \geq 2$ for which
		\begin{enumerate}
			\item[a)] $K_v$ has an open euler walk.
			\item[b)] $K_v$ has an closed euler walk.
			\item[c)] $K_v$ has an open hamilton walk.
			\item[d)] $K_v$ has an closed hamilton walk.
		\end{enumerate}
		\item[3] Explain why each drawing in Figure 155 are either bad or good puzzles.
		\item[4] Let $C$ be a graph with $v = 64$, its vertices corresponding to the squares of a chess board. Let two vertices of $C$ be joined by an edge whenever a knight can go from one of the corresponding squares to the other in one move. Does $C$ have an euler walk? (You don't have to draw $C$ to answer.)
		\item[5] Prove that $C$ from the previous exercise has a closed hamilton walk. Such a walk is called a ``knight's tour" by puzzle enthusiasts.
	\end{enumerate}
\end{document}